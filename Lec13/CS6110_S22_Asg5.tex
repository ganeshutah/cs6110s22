
% --------------------------------------------------------------
% This is all preamble stuff that you don't have to worry about.
% Head down to where it says "Start here"
% --------------------------------------------------------------

\documentclass[11pt]{article}

\usepackage{bera}
%\renewcommand{\familydefault}{\rmfamily}

\usepackage{graphicx,url}
\usepackage{proof}
\usepackage{framed}
\usepackage{etaremune}

\usepackage[margin=1in]{geometry}
\usepackage{amsmath,amsthm,amssymb,amsfonts}
\usepackage{paralist}
\thispagestyle{empty}

% 1. To get version suitable for students to populate,
%    remove the contents of the \ignoreSoln{..body..}
%
% 2. To get a version suitable for generating PDF 
%    without solutions, remove the #1 below
%
% 3. To generate solutions, keep the #1 below
%
% 4. Assigned grader fills \ignoreSoln{..body..}
%    and also provides his/her feedback to student
%    and policy followed for point deduction
%    So design policy before grading begins.

\newcommand{\ignoreSoln}[1]{#1}   
%\newcommand{\ignoreModel}[1]{#1} 


\newcommand{\bigset}[2]{\big\{\;#1\;:\;#2\;\big\}}
\newcommand{\N}{\mathbb{N}}
\newcommand{\Z}{\mathbb{Z}}
\newcommand{\R}{\mathbb{R}}
\newcommand{\Np}{\mathbb{N^{+}}}

\newenvironment{theorem}[2][Theorem]{\begin{trivlist}
\item[\hskip \labelsep {\bfseries #1}\hskip \labelsep {\bfseries #2.}]}{\end{trivlist}}
\newenvironment{lemma}[2][Lemma]{\begin{trivlist}
\item[\hskip \labelsep {\bfseries #1}\hskip \labelsep {\bfseries #2.}]}{\end{trivlist}}
\newenvironment{exercise}[2][Exercise]{\begin{trivlist}
\item[\hskip \labelsep {\bfseries #1}\hskip \labelsep {\bfseries #2.}]}{\end{trivlist}}
\newenvironment{reflection}[2][Reflection]{\begin{trivlist}
\item[\hskip \labelsep {\bfseries #1}\hskip \labelsep {\bfseries #2.}]}{\end{trivlist}}
\newenvironment{proposition}[2][Proposition]{\begin{trivlist}
\item[\hskip \labelsep {\bfseries #1}\hskip \labelsep {\bfseries #2.}]}{\end{trivlist}}
\newenvironment{corollary}[2][Corollary]{\begin{trivlist}
\item[\hskip \labelsep {\bfseries #1}\hskip \labelsep {\bfseries #2.}]}{\end{trivlist}}

\DeclareMathSizes{14}{14}{14}{14}

\begin{document}

% --------------------------------------------------------------
%                         Start here
% --------------------------------------------------------------

%\renewcommand{\qedsymbol}{\filledbox}
\newlength{\minpagw}
\settowidth{\minpagw}{\hspace{40em}}

\begin{center}
\begin{large}
  CS 6110, Spring 2022, Assignment 5  \\
  Given 2/22/22 -- Due 3/3/22 by 11:59 pm via your Github 
  \ \\
%  \ \\  
    {  {\Large\bf NAME: } \hfill {\Large\bf UNID: }\hspace{4cm} }
          \ \\
\end{large}

\end{center}

\noindent{\bf CHANGES:\/} {\bf Please look for lines beginning with underlined words when they are made.}
         {\tiny none yet.}

         \noindent {\bf Answering, Submission:\/}
         Have these on your private Github:
         a folder Asg4/ containing your submission, which in detail comprises:
         \begin{compactitem}
         \item A clear README.md describing your files.
         \item Files that you ran + documentation (can be integrated in one place).
         \item A high level summary of your cool findings + insights + learning -- briefly reported in
           a nicely bulletted fashion in your PDF submission.
         \end{compactitem}

         \noindent {\bf Start Early, Ask Often!}
Orientation videos and further help will be available (drop a note anytime
on Piazza for help).

I encourage students constructing answers jointly! {\em But that does not
mean copy solutions, but discuss the question plus surrounding issues.}

\begin{enumerate}

%- 1 ----------------------------------------------------------------
\item (10 points)
  Read and summarize (in one page) Jackson's article on Alloy
  from \url{https://cacm.acm.org/magazines/2019/9/238969-alloy/fulltext#R1}
  in exactly two pages. Please capture all the details highlighted there.
  Try to write a good summary for your own understanding!

  
\begin{minipage}{\minpagw}
  \fbox{%
    \parbox{\linewidth}{%
      Your

      Answer

      Here
    }%
  }%
\end{minipage}  
  

%- 2 ----------------------------------------------------------------
\item (25 points)
  Take the tutorial on
  \url{https://www.doc.ic.ac.uk/project/examples/2007/271j/suprema_on_alloy/Web/index.php}
  and take the assessment questionnaire at the end! Try to score a perfect score (it gives you
  hints and retry opportunities).
  %
  {\em Record that you did the above in about two pages -- capture screenshots, sessions, etc, convincing
    me that you took the assessment.} Report your assessment score as a screenshot! (Get a perfect 10/10.)

%- 3 ----------------------------------------------------------------
\item (40 points)
  First, deep-read the Alloy ``Red book'' linked on the class website
  from the beginning till Page 80.
  Summarize the concepts you find on these pages in four pages.
  {\em I want to see a good summary.}
  \item[] Then do the work below.
    Three models for trees are below (these
  are from
  \url{https://stackoverflow.com/questions/41707898/is-there-a-better-alloy-model-of-a-tree}).
  %
  I saw the beginnings of Costello's tree solution (which Daniel Jackson, author of Alloy, improves upon) and wrote
  my own solution.
  %
  Thus, there are three tree models below.
  %
  Study them and do the problems stated below them:
  \begin{scriptsize}
\begin{verbatim}

/*-- BEGIN: common to all three tree definitions --*/

sig Node {
  tree: set Node
}

one sig root extends Node {} // root is a subset of Node

/*-- END : common to all three tree definitions --*/

pred GGTree{
 no n : Node | root in n.^tree // Q-1
 --
 no n : Node | n in n.^tree    // Q-2
 --
 all n : Node - root | n in root.^tree // Q-3
 -- 
 all n : Node |
 all disj n1, n2 : n.tree |    // Q-4
   no (n1.*tree & n2.*tree) 
}
  \end{verbatim}
  \end{scriptsize}

\begin{itemize}
\item[] Q-1,2: Express what these definitions are saying, in English.
Why can't these definitions use \verb|*|? What happens if you do that?
 Generate
a few tree instances (models) and show what happens if you use \verb|*| for Q-1 and Q-2?.
%
{\em To generate instances,} you must remove the "pred" temporarily, and express its
contents as signature facts.
This is what you would be doing initially in your work on creating a model, anyhow.
Then follow such a modified definition with a
\verb|run {} for 3|,
or
\verb|run {} for exactly 10 node|
or
similar notations (consult the ``Red'' Alloy book mentioned on the class website for details).
Then execute, and the ``run'' gives you the instances I'm looking for.

\item[] Q-3: Express what this definition is saying, in English.
Why do we need \verb|Node - root|? Why can't you use \verb|Node| here?
Again show by generating instances saying what happens if you do that.

\item[] Q-4: What is this saying in English?
What happens if you leave out \verb|disj|?
Show by generating some instances.

\item Now put the {\tt pred} definitions back. So you have three "pred" definitions as presented below.

\begin{scriptsize}
\begin{verbatim}

pred GGTree{
 no n : Node | root in n.^tree
 --
 no n : Node | n in n.^tree
 --
 all n : Node - root | n in root.^tree
 -- 
 all n : Node |
 all disj n1, n2 : n.tree |  
   no (n1.*tree & n2.*tree) 
}


pred DJTree {
    Node in root.*tree // all reachable
    no iden & ^tree // no cycles
    tree in  Node lone -> Node // Q-5
    }

pred CostelloTree {
    // No node above root (no node maps to root)
    no tree.root
    // Can reach all nodes from root                
    all n: Node - root | n in root.^tree
    // No node maps to itself (irreflexive) 
    no iden & tree
    // No cycles                    
    no n: Node | Node in n.^tree
    // All nodes are distinct (injective)           
    tree.~tree in iden -- need this
}
\end{verbatim}
\end{scriptsize}

\item[] Explain the line tagged Q-5, above. You will find the constraints of signatures explained
  in the ``Red'' book. Look around Page 78, Section 3.6.3.

The goal is to show the equivalences below:

\begin{enumerate}
\item Issue \verb|check {GGTree iff DJTree} for 5| and see if the definitions are equivalent.

\item Just break the Q-1 to have \verb|*| and not the correct \verb|^|. Do the ``\verb|iff|''  check above.
Do you get an understandable counterexample? Describe it.

\item Issue \verb|check {CostelloTree iff DJTree} for 5| and see if the definitions are equivalent.
\end{enumerate}

\end{itemize}
  
\begin{minipage}{\minpagw}
  \fbox{%
    \parbox{\linewidth}{%
      Your

      Answer

      Here
    }%
  }%
\end{minipage}  


%- 2 ----------------------------------------------------------------
\item (25 points) Write your own quicksort in C for a character array of 8
  locations, and subject it to KLEE tests, as illustrated in
  Lecture 13. Write a bug-free version. Then put one bug that does not sort correctly.
  Put an assertion at the end for a mis-sorted outcome. Hit that assertion using a KLEE-test.

  \begin{minipage}{\minpagw}
  \fbox{%
    \parbox{\linewidth}{%
      Your

      Answer

      Here
    }%
  }%
\end{minipage}  


  
%- end ----------------------------------------------------------------  


\end{enumerate}

\end{document}
