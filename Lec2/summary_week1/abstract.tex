
 An introduction to the kinds of problems verified and solved. The highlights are: TimSort was mentioned in Lecture-2 as an example where a widely used sorting method proved to be buggy. Most of the time was spent in reviewing the foundations. For instance: (1) Verification splits itself into automata-theoretic methods and Boolean Methods; (2) Boolean methods are able to handle large problems despite Satisfiability being NP-complete.
 
 Lecture-2's focus was on Tseitin transformation. It also served as an excuse to review Boolean Algebra. Then we illustrated how the coding of a sorting problem in Promela looks like.
 
 Assignment-1 has been issued, but due in a week.
 The pattern thereafter will be: due in five days.